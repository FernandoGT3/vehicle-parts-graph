\documentclass[aspectratio=169]{beamer}

% --- Pacotes e Configurações ---
\usepackage[utf8]{inputenc}
\usepackage[portuguese]{babel}
\usepackage{graphicx}
\usepackage{booktabs} % Para tabelas bonitas
\usepackage{amsmath}
\usepackage{ragged2e} % Para justificar texto
\usepackage{tikz}

% --- Tema Visual ---
% Sugestão de tema profissional e limpo
\usetheme{Madrid}
\usecolortheme{beaver} % Esquema de cores vermelho/cinza (sóbrio)
\setbeamercovered{transparent} % Transparência em itens futuros

% --- Metadados ---
\title[Resiliência na Cadeia Automotiva]{Análise Topológica e Resiliência na Cadeia de Suprimentos Automotiva}
\subtitle{Uma Abordagem via Teoria dos Grafos}
\author{Caio, Eduardo, Felipe, Luiz, Matheus (Grupo Esperança)}
\institute{Universidade Federal de São Paulo}
\date{\today}

\begin{document}

% --- Slide 1: Capa ---
\begin{frame}
    \titlepage
\end{frame}

% --- Slide 2: Agenda ---
\begin{frame}{Agenda}
    \tableofcontents
\end{frame}

% =================================================================
% SEÇÃO 1: INTRODUÇÃO
% =================================================================
\section{Introdução e Contexto}

\begin{frame}{Contexto da Indústria Automotiva}
    \begin{columns}
        \column{0.6\textwidth}
        \justifying
        Nas últimas duas décadas, o setor passou por uma transformação radical visando redução de custos.
        \vspace{0.5cm}
        \textbf{Engenharia de Plataformas:}
        \begin{itemize}
            \item Abandono do projeto de "unidades isoladas".
            \item Adoção de "kits de construção" modulares.
            \item \textbf{Exemplos:} VW (MQB), Toyota (TNGA), Stellantis (CMP).
            \item Compartilhamento massivo entre marcas do mesmo conglomerado.
        \end{itemize}

        \column{0.4\textwidth}
        \begin{alertblock}{O Problema da Centralização}
            Embora eficiente, cria interdependências invisíveis.
            Uma falha trivial (ex: microchip) paralisa múltiplas marcas simultaneamente.
        \end{alertblock}
    \end{columns}
\end{frame}

\begin{frame}{Objetivos do Trabalho}
    Buscamos "desvendar" a teia de conexões respondendo a 5 perguntas chave:
    \vspace{0.5cm}
    \begin{enumerate}
        \item \textbf{Topologia:} Qual a estrutura real das alianças industriais?
        \item \textbf{Vulnerabilidade:} Onde estão os pontos únicos de falha (\textit{Single Points of Failure})?
        \item \textbf{Resiliência:} Quão robusta é a malha frente a colapsos?
        \item \textbf{Segmentação:} O compartilhamento respeita a divisão Premium vs Economy?
        \item \textbf{Otimização:} É possível prever demandas futuras via rede?
    \end{enumerate}
\end{frame}

% =================================================================
% SEÇÃO 2: METODOLOGIA
% =================================================================
\section{Metodologia}

\begin{frame}{Modelagem Matemática}
    \begin{block}{Grafo Bipartido $G = (U, V, E)$}
        Modelamos o sistema com dois conjuntos disjuntos:
        \begin{itemize}
            \item $U = \{u_1, ..., u_n\}$: Conjunto de \textbf{36 Veículos}.
            \item $V = \{v_1, ..., v_m\}$: Conjunto de \textbf{48 Peças/Sistemas}.
        \end{itemize}
        Aresta $(u_i, v_j)$ existe se o veículo utiliza a peça.
    \end{block}

    \vspace{0.5cm}
    \textbf{Projeção Unimodal:}
    Realizamos uma projeção onde $w_{xy}$ (peso) representa o número de peças compartilhadas entre dois carros.
\end{frame}

\begin{frame}{Visualização da Rede Bipartida}
    \begin{figure}
        \centering
        % Ajuste a escala conforme necessário
        \includegraphics[height=0.75\textheight]{../assets/fig1_network.png} 
        \caption{Alta densidade de conexões convergindo para nós centrais (hubs).}
    \end{figure}
\end{frame}

% =================================================================
% SEÇÃO 3: ANÁLISE TOPOLÓGICA
% =================================================================
\section{Análise da Topologia}

\begin{frame}{Detecção de Comunidades (Modularidade)}
    \begin{columns}
        \column{0.5\textwidth}
        Utilizando \textit{Greedy Modularity Maximization}, detectamos 3 comunidades que refletem as alianças globais:
        \vspace{0.5cm}
        \begin{itemize}
            \item \textbf{Família 1:} VW Group (VW, Audi, Porsche).
            \item \textbf{Família 2:} Renault-Nissan + Stellantis.
            \item \textbf{Família 3:} BMW, Mercedes, Toyota, Honda, Ford, Volvo.
        \end{itemize}

        \column{0.5\textwidth}
        \begin{figure}
            \centering
            \includegraphics[width=\textwidth]{../assets/fig2_clusters.png}
            \caption{Clusters Estratégicos Detectados.}
        \end{figure}
    \end{columns}
\end{frame}

% \begin{frame}{Coeficiente de Clustering}
%     \begin{columns}
%         \column{0.4\textwidth}
%         \justifying
%         Calculamos o coeficiente de clustering local para avaliar a maturidade das plataformas.
%         \vspace{0.5cm}
%         \begin{itemize}
%             \item Valores altos indicam vizinhos conectados entre si.
%             \item Característica de plataformas maduras com forte integração interna.
%         \end{itemize}

%         \column{0.6\textwidth}
%         \begin{figure}
%             \centering
%             \includegraphics[width=\textwidth]{../assets/fig10_local_clustering.png}
%             \caption{Clustering Local por Veículo.}
%         \end{figure}
%     \end{columns}
% \end{frame}

\begin{frame}{Assortatividade e Mistura de Segmentos}
    \begin{columns}
        \column{0.5\textwidth}
        \textbf{Coeficiente de Assortatividade ($r \approx -0.008$):}
        \vspace{0.5cm}
        \begin{itemize}
            \item Valor próximo de zero indica \textbf{Homogeneização Tecnológica}.
            \item A matriz de mistura mostra distribuição equilibrada.
            \item Carros \textit{Premium} e \textit{Economy} compartilham a mesma "alma mecânica".
        \end{itemize}

        \column{0.5\textwidth}
        \begin{figure}
            \centering
            \includegraphics[width=0.9\textwidth]{../assets/fig9_mixing_matrix.png}
            \caption{Matriz de Mistura (Mixing Matrix).}
        \end{figure}
    \end{columns}
\end{frame}

\begin{frame}{Similaridade de Jaccard}
    \begin{columns}
        \column{0.4\textwidth}
        \textbf{Índice de Jaccard:}
        \begin{equation*}
            J(A,B) = \frac{|A \cap B|}{|A \cup B|}
        \end{equation*}
        \vspace{0.3cm}
        O Heatmap revela "gêmeos de plataforma":
        \begin{itemize}
            \item Blocos diagonais confirmam os clusters.
            \item Identifica veículos com base tecnológica idêntica.
        \end{itemize}

        \column{0.6\textwidth}
        \begin{figure}
            \centering
            \includegraphics[width=\textwidth]{../assets/fig7_jaccard_heatmap.png}
            \caption{Heatmap de Similaridade.}
        \end{figure}
    \end{columns}
\end{frame}

% =================================================================
% SEÇÃO 4: VULNERABILIDADE
% =================================================================
\section{Vulnerabilidade e Riscos}

\begin{frame}{Distribuição de Graus (Lei de Potência)}
    \begin{columns}
        \column{0.4\textwidth}
        A rede exibe comportamento \textit{Scale-Free}.
        \vspace{0.5cm}
        \begin{itemize}
            \item Linearidade no gráfico log-log.
            \item Poucos nós possuem conexões desproporcionais.
            \item Estrutura típica de sistemas vulneráveis a ataques direcionados.
        \end{itemize}

        \column{0.6\textwidth}
        \begin{figure}
            \centering
            \includegraphics[width=\textwidth]{../assets/fig8_degree_distribution.png}
            \caption{Distribuição de Graus.}
        \end{figure}
    \end{columns}
\end{frame}

\begin{frame}{Identificação de Hubs (Infraestrutura Crítica)}
    Poucos nós (Hubs) concentram o risco sistêmico.
    
    \begin{table}
        \centering
        \footnotesize
        \begin{tabular}{clccc}
        \toprule
        \textbf{Rank} & \textbf{Componente (Hub)} & \textbf{Grau} & \textbf{Eigen} & \textbf{Betweenness} \\
        \midrule
        1 & Sistema ABS Bosch & 28 & 0.350 & 0.485 \\
        2 & Suspensão Multilink & 9 & 0.285 & 0.112 \\
        3 & Turbocompressor KKK & 8 & 0.245 & 0.098 \\
        4 & Transmissão DSG DQ250 & 6 & 0.198 & 0.067 \\
        5 & Motor EA888 2.0T & 4 & 0.165 & 0.045 \\
        \bottomrule
        \end{tabular}
        \caption{Top 5 Componentes Críticos}
    \end{table}
\end{frame}

\begin{frame}{Comparação Multicritério de Centralidade}
    \begin{figure}
        \centering
        \includegraphics[height=0.75\textheight]{../assets/fig12_centrality_comparison.png}
        \caption{Comparação: Grau vs Intermediação e Grau vs Autovetor. O ABS Bosch destaca-se em todas as dimensões.}
    \end{figure}
\end{frame}

\begin{frame}{Visualização dos Hubs}
    \begin{figure}
        \centering
        \includegraphics[height=0.75\textheight]{../assets/fig3_hubs.png}
        \caption{Disparidade de conectividade: O "Sistema ABS Bosch" como Super-Hub.}
    \end{figure}
\end{frame}

\begin{frame}{Decomposição K-Core: O Núcleo Estável}
    \begin{columns}
        \column{0.5\textwidth}
        \justifying
        A análise revelou um núcleo máximo com \textbf{$k_{max} = 27$}.
        \vspace{0.5cm}
        Os veículos neste núcleo formam a espinha dorsal do mercado, compartilhando peças com pelo menos outros 27 modelos.
        Isso demonstra a profundidade da interconexão da indústria.

        \column{0.5\textwidth}
        \begin{figure}
            \centering
            \includegraphics[width=\textwidth]{../assets/fig6_kcore.png}
            \caption{Camadas da Rede (K-Shells).}
        \end{figure}
    \end{columns}
\end{frame}

\begin{frame}{Simulação de Colapso em Cascata (Stress Test)}
    \begin{columns}
        \column{0.4\textwidth}
        Simulamos a falha sequencial dos 5 maiores hubs.
        
        \vspace{0.3cm}
        \begin{alertblock}{Resultado Crítico}
            A rede sofre uma \textbf{transição de fase abrupta}.
            A perda de apenas 1 hub fragmenta a rede massivamente (perda de 19\%).
        \end{alertblock}

        \column{0.6\textwidth}
        \begin{figure}
            \centering
            \includegraphics[width=\textwidth]{../assets/fig4_resilience.png}
            \caption{Curva de degradação da rede.}
        \end{figure}
    \end{columns}
\end{frame}

% =================================================================
% SEÇÃO 5: APLICAÇÕES PRÁTICAS
% =================================================================
\section{Aplicações Práticas}

% \begin{frame}{Algoritmo de Recomendação e Padronização}
%     Utilizando Filtragem Colaborativa nos clusters, identificamos oportunidades de padronização (ou erros de cadastro).
%     \vspace{0.5cm}
%     \begin{table}
%         \centering
%         \small
%         \begin{tabular}{ll}
%         \toprule
%         \textbf{Veículo Alvo} & \textbf{Sugestão (Baseado no Cluster)} \\
%         \midrule
%         Honda CR-V & Sistema ABS Bosch \\
%         Mercedes E-Class (W213) & Sistema ABS Bosch \\
%         Ford Focus & Airbag de Cortina \\
%         Ford Kuga & Sistema ABS Bosch \\
%         \bottomrule
%         \end{tabular}
%     \end{table}
% \end{frame}

\begin{frame}{Backbone da Indústria (MST)}
    \begin{figure}
        \centering
        \includegraphics[height=0.75\textheight]{../assets/fig5_backbone.png}
        \caption{Árvore Geradora Máxima: A infraestrutura mínima conectada.}
    \end{figure}
\end{frame}

\begin{frame}{Eficiência de Estoque}
    Quantificamos a motivação econômica da estratégia de plataformas:
    
    \vspace{0.5cm}
    \begin{center}
        \huge
        Redução de Complexidade: \\
        \textbf{\textcolor{red}{65.7\%}}
    \end{center}
    
    \vspace{0.5cm}
    \normalsize
    Demanda agregada de 140 partes reduzida para um inventário real de 48 peças únicas.
\end{frame}

% =================================================================
% SEÇÃO 6: CONCLUSÃO
% =================================================================
\section{Conclusão}

\begin{frame}{Conclusões e Limitações}
    \begin{block}{Conclusão Principal}
        Topologia otimizada para \textbf{Eficiência}, mas estruturalmente \textbf{Frágil}.
    \end{block}

    \begin{itemize}
        \item \textbf{Mundo Pequeno:} Facilita difusão de falhas.
        \item \textbf{Risco Sistêmico:} Dependência extrema de Super-Hubs (ex: ABS Bosch).
        \item \textbf{Colapso:} Falha de 1 componente pode paralisar 77\% da frota.
    \end{itemize}
    
    \vspace{0.3cm}
    \textbf{Limitações:}
    \begin{itemize}
        \item Dataset pequeno (36 veículos) e estático.
        \item Pesos uniformes nas arestas.
    \end{itemize}
\end{frame}

\begin{frame}{Código - Menu}
    \begin{figure}
        \centering
        \includegraphics[height=0.6\textheight]{../assets/menu.png}
        \caption{Menu interativo construído durante a disciplina.}
    \end{figure}

\end{frame}


\begin{frame}{Arquivos de Dados I}
    \begin{figure}
        \centering
        \includegraphics[height=0.6\textheight]{../assets/dados1.png}
        \caption{Arquivo de entrada do Grafo - data.py}
    \end{figure}

\end{frame}
\begin{frame}{Arquivos de Dados II}

    \begin{figure}
        \centering
        \includegraphics[height=0.6\textheight]{../assets/dados2.png}
        \caption{Arquivo de entrada do Grafo - data.py}
    \end{figure}

\end{frame}

\end{document}