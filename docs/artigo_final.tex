\documentclass[12pt, a4paper]{article}
\usepackage[utf8]{inputenc}
\usepackage[portuguese]{babel}
\usepackage{graphicx}
\usepackage{geometry}
\usepackage{booktabs}
\usepackage{hyperref}
\usepackage{float}
\usepackage{caption}
\usepackage{amsmath}
\usepackage{setspace}

\geometry{a4paper, total={170mm,257mm}, left=25mm, top=25mm, right=25mm, bottom=25mm}
\onehalfspacing

\title{\textbf{Análise Topológica e Resiliência na Cadeia de Suprimentos Automotiva: Uma Abordagem via Teoria dos Grafos}}
\author{Caio, Eduardo, Felipe, Luiz, Matheus (Grupo Esperança)}
\date{\today}

\begin{document}

\maketitle

\begin{abstract}
\noindent Este estudo propõe uma modelagem matemática baseada em Teoria dos Grafos Complexos para analisar a estrutura oculta de compartilhamento de componentes entre 36 modelos de veículos modernos e 48 componentes/sistemas. A indústria automotiva contemporânea caracteriza-se pela adoção massiva de plataformas modulares, estratégia que visa eficiência econômica mas que potencialmente introduz riscos sistêmicos. Utilizando métricas avançadas — incluindo detecção de comunidades por modularidade, análise multicritério de centralidade (grau, intermediação e autovetor), decomposição K-Core, coeficientes de assortatividade e similaridade de Jaccard — investigamos o trade-off entre eficiência e robustez. Nossos resultados demonstram que, embora a estratégia de plataformas reduza os custos de complexidade de estoque em aproximadamente 65.7\%, ela cria uma topologia de rede frágil, dependente de poucos 'super-fornecedores'. Simulações de falha em cascata revelam que o colapso de um único componente crítico pode paralisar até 77\% da frota analisada, evidenciando a necessidade urgente de diversificação na cadeia de suprimentos.
\end{abstract}

\newpage
\tableofcontents
\newpage

\section{Introdução}

\subsection{Contexto da Indústria Automotiva}
Nas últimas duas décadas, o setor automotivo global passou por uma transformação radical impulsionada pela necessidade de redução de custos e aceleração do tempo de desenvolvimento de novos produtos. A resposta técnica e gerencial para esses desafios foi a adoção da chamada "Engenharia de Plataformas". Montadoras como o Grupo Volkswagen (com a plataforma MQB), Toyota (com a TNGA), Stellantis (plataformas CMP e EMP2) e Renault-Nissan (CMF) deixaram de projetar carros como unidades isoladas para projetar "kits de construção" modulares, onde motores, transmissões, sistemas elétricos e até partes estruturais são compartilhados entre dezenas de modelos, muitas vezes de marcas diferentes sob o mesmo conglomerado.

\subsection{O Problema da Centralização}
Embora economicamente vantajosa, essa estratégia cria uma rede de interdependências profundas e muitas vezes invisíveis. A falha na produção de um componente aparentemente trivial (como um microchip ou um sistema de freios) não afeta mais apenas um modelo, mas pode paralisar linhas de produção de múltiplas marcas simultaneamente. Este fenômeno foi observado globalmente durante a crise dos semicondutores (2020-2023).

\subsection{Objetivos do Trabalho}
O objetivo deste trabalho é "desvendar" essa teia de conexões utilizando ferramentas quantitativas da Ciência das Redes. Buscamos responder:
\begin{enumerate}
    \item \textbf{Topologia:} Qual é a estrutura real das alianças industriais? Os grupos se comportam como esperado?
    \item \textbf{Vulnerabilidade:} Onde estão os pontos únicos de falha (Single Points of Failure)?
    \item \textbf{Resiliência:} Quão robusta é a malha produtiva frente a colapsos de fornecedores chave?
    \item \textbf{Segmentação:} A estratégia de compartilhamento respeita a segmentação de mercado (Premium vs Economy)?
    \item \textbf{Otimização:} É possível prever demandas futuras e identificar oportunidades de padronização?
\end{enumerate}

\section{Metodologia}

\subsection{Modelagem Matemática}
O sistema foi modelado como um \textbf{Grafo Bipartido} $G = (U, V, E)$, onde os dois conjuntos disjuntos de vértices são:
\begin{itemize}
    \item $U = \{u_1, u_2, ..., u_n\}$: O conjunto de 36 Veículos.
    \item $V = \{v_1, v_2, ..., v_m\}$: O conjunto de 48 Peças/Sistemas.
\end{itemize}
Uma aresta $(u_i, v_j) \in E$ existe se, e somente se, o veículo $u_i$ utiliza o componente $v_j$.

Para analisar as relações diretas entre os veículos, realizamos uma \textbf{Projeção Unimodal} ponderada $P$, onde os nós são apenas os veículos, e o peso da aresta $w_{xy}$ entre dois carros $x$ e $y$ representa o número de peças que eles compartilham.

\subsection{Implementação Computacional}
O sistema foi implementado em Python utilizando a biblioteca NetworkX para operações de grafos e Matplotlib para visualizações. A arquitetura modular do código inclui:
\begin{itemize}
    \item \texttt{data.py}: Dataset com veículos, peças e relações
    \item \texttt{graph\_ops.py}: Operações de grafos e métricas avançadas
    \item \texttt{visualizer.py}: Geração de visualizações de alta resolução
    \item \texttt{report\_generator.py}: Geração automatizada de relatórios
\end{itemize}

\subsection{Métricas de Centralidade Multicritério}
Diferentemente de análises simplificadas, utilizamos três métricas complementares de centralidade:
\begin{enumerate}
    \item \textbf{Centralidade de Grau (Degree Centrality):} Mede o volume absoluto de conexões — economia de escala.
    \item \textbf{Centralidade de Intermediação (Betweenness Centrality):} Identifica gargalos estratégicos — risco de contágio.
    \item \textbf{Centralidade de Autovetor (Eigenvector Centrality):} Mede a influência baseada na importância dos vizinhos.
\end{enumerate}

\begin{figure}[H]
    \centering
    \includegraphics[width=0.95\textwidth]{../assets/fig1_network.png}
    \caption{Representação visual da rede bipartida Veículo-Peça. Nós azuis representam veículos; nós laranja representam peças. A alta densidade de conexões convergindo para nós centrais evidencia a presença de hubs.}
    \label{fig:network}
\end{figure}

\section{Análise da Topologia da Rede}

\subsection{Detecção de Comunidades e Modularidade}
Uma das questões centrais é verificar se o compartilhamento de peças respeita as fronteiras logísticas das grandes montadoras. Para isso, aplicamos o algoritmo de \textit{Greedy Modularity Maximization}. Este algoritmo particiona a rede em comunidades de forma a maximizar a densidade de arestas dentro dos grupos em relação às arestas entre grupos.

O algoritmo detectou automaticamente \textbf{3 comunidades distintas}. A inspeção visual (Figura \ref{fig:clusters}) e analítica revela que estes grupos correspondem quase perfeitamente às grandes alianças globais:
\begin{itemize}
    \item \textbf{Família 1:} VW Group (VW, Audi, Porsche)
    \item \textbf{Família 2:} Renault-Nissan Alliance + Stellantis
    \item \textbf{Família 3:} BMW, Mercedes, Toyota, Honda, Ford, Volvo
\end{itemize}

\begin{figure}[H]
    \centering
    \includegraphics[width=0.95\textwidth]{../assets/fig2_clusters.png}
    \caption{Grafo projetado (veículo-veículo) onde as cores indicam os clusters detectados automaticamente. A espessura das arestas representa o número de peças compartilhadas. A forte clusterização confirma que a engenharia de plataformas cria silos de integração.}
    \label{fig:clusters}
\end{figure}

\subsection{Coeficiente de Clustering e Transitividade}
Para avaliar a maturidade das plataformas, calculamos o coeficiente de clustering local para cada veículo (Figura \ref{fig:clustering}). Valores altos indicam que os vizinhos de um nó também são conectados entre si, característica de plataformas bem definidas.

\begin{figure}[H]
    \centering
    \includegraphics[width=0.95\textwidth]{../assets/fig10_local_clustering.png}
    \caption{Coeficiente de clustering local por veículo. Veículos com alto clustering pertencem a plataformas maduras com forte integração interna.}
    \label{fig:clustering}
\end{figure}

\subsection{Assortatividade e Mistura de Segmentos}
Investigamos se existe segregação tecnológica entre carros de luxo (Premium) e carros populares (Economy). O coeficiente de assortatividade calculado foi $r \approx -0.008$.

A Figura \ref{fig:mixing} apresenta a matriz de mistura, que mostra a proporção de conexões entre segmentos. O valor próximo de zero revela uma \textbf{homogeneização tecnológica}: componentes críticos tornaram-se commodities genéricas, utilizadas indistintamente por marcas Premium e generalistas.

\begin{figure}[H]
    \centering
    \includegraphics[width=0.7\textwidth]{../assets/fig9_mixing_matrix.png}
    \caption{Matriz de mistura por segmento de mercado. A distribuição equilibrada indica que veículos Premium e Economy compartilham componentes de forma cruzada, sem segregação significativa.}
    \label{fig:mixing}
\end{figure}

\subsection{Similaridade de Jaccard}
Para normalizar o compartilhamento pelo tamanho total dos conjuntos de peças, calculamos o Índice de Similaridade de Jaccard:
\begin{equation}
    J(A,B) = \frac{|A \cap B|}{|A \cup B|}
\end{equation}

O heatmap (Figura \ref{fig:jaccard}) revela os pares de veículos com maior similaridade, permitindo identificar "gêmeos de plataforma" que compartilham praticamente a mesma base tecnológica.

\begin{figure}[H]
    \centering
    \includegraphics[width=0.95\textwidth]{../assets/fig7_jaccard_heatmap.png}
    \caption{Heatmap de similaridade de Jaccard entre todos os pares de veículos. Células mais escuras indicam maior sobreposição de componentes. Blocos diagonais confirmam os clusters identificados.}
    \label{fig:jaccard}
\end{figure}

\section{Análise de Vulnerabilidade e Riscos}

\subsection{Identificação de Hubs (Infraestrutura Crítica)}
A análise multicritério de centralidade no grafo bipartido permite identificar quais peças são os pilares da indústria. A Figura \ref{fig:degree_dist} mostra a distribuição de graus da rede projetada, revelando uma \textbf{distribuição de Lei de Potência (Power Law)} onde poucos nós possuem conexões desproporcionais.

\begin{figure}[H]
    \centering
    \includegraphics[width=0.95\textwidth]{../assets/fig8_degree_distribution.png}
    \caption{Distribuição de graus em escala linear (esquerda) e log-log (direita). A linearidade aproximada no gráfico log-log confirma o comportamento scale-free da rede, típico de sistemas com hubs dominantes.}
    \label{fig:degree_dist}
\end{figure}

A Tabela \ref{tab:hubs} lista os componentes mais críticos. O \textbf{Sistema ABS Bosch} atua como um \textit{Super-Hub} — sua onipresença representa um risco sistêmico. A Figura \ref{fig:centrality_comp} compara as três métricas de centralidade, permitindo identificar componentes que são simultaneamente volumosos (alto grau), influentes (alto autovetor) e gargalos (alta intermediação).

\begin{table}[H]
    \centering
    \caption{Ranking Multicritério de Criticidade (Top 5)}
    \label{tab:hubs}
    \begin{tabular}{clccc}
    \toprule
    Rank & Componente (Hub) & Grau & Importância (Eigen) & Gargalo (Betw.) \\
    \midrule
    1 & Sistema ABS Bosch & 28 & 0.350 & 0.485 \\
    2 & Suspensão Multilink & 9 & 0.285 & 0.112 \\
    3 & Turbocompressor KKK & 8 & 0.245 & 0.098 \\
    4 & Transmissão DSG DQ250 & 6 & 0.198 & 0.067 \\
    5 & Motor EA888 2.0T & 4 & 0.165 & 0.045 \\
    \bottomrule
    \end{tabular}
\end{table}

\begin{figure}[H]
    \centering
    \includegraphics[width=0.95\textwidth]{../assets/fig12_centrality_comparison.png}
    \caption{Comparação multicritério de centralidades. Esquerda: Grau vs Intermediação. Direita: Grau vs Autovetor. O tamanho dos pontos representa o grau; a cor representa a terceira métrica. O Sistema ABS Bosch destaca-se em todas as dimensões.}
    \label{fig:centrality_comp}
\end{figure}

\begin{figure}[H]
    \centering
    \includegraphics[width=0.85\textwidth]{../assets/fig3_hubs.png}
    \caption{As 15 peças mais conectadas da indústria — ranking por número de veículos dependentes.}
    \label{fig:hubs}
\end{figure}

\subsection{Decomposição K-Core: O Núcleo Estável}
Para entender a profundidade da interconexão, realizamos a \textbf{decomposição K-Core} (Figura \ref{fig:kcore}). O processo revela a estrutura em camadas da indústria:
\begin{itemize}
    \item \textbf{Núcleo máximo} ($k_{max} = 27$): Veículos que compartilham peças com pelo menos outros 27 modelos.
    \item \textbf{Periferia} (cascas externas): Modelos de nicho ou com tecnologias proprietárias.
\end{itemize}

\begin{figure}[H]
    \centering
    \includegraphics[width=0.95\textwidth]{../assets/fig6_kcore.png}
    \caption{Decomposição K-Core do grafo projetado. Nós mais escuros (pretos) pertencem ao núcleo mais profundo e interconectado; nós mais claros (amarelos) estão na periferia.}
    \label{fig:kcore}
\end{figure}

\subsection{Simulação de Colapso em Cascata}
Realizamos um \textbf{Stress Test} simulando a falha sequencial dos 5 maiores hubs. A métrica de controle foi o tamanho do Componente Gigante Conectado (GCC). A Figura \ref{fig:resilience} mostra a curva de degradação.

\begin{figure}[H]
    \centering
    \includegraphics[width=0.85\textwidth]{../assets/fig4_resilience.png}
    \caption{Curva de resiliência da rede. A queda acentuada após a remoção do primeiro hub é característica de redes Scale-Free sob ataque direcionado. A rede perde 19\% de conectividade com a falha de um único componente.}
    \label{fig:resilience}
\end{figure}

\begin{table}[H]
    \centering
    \caption{Impacto da Falha Sequencial de Fornecedores}
    \begin{tabular}{ccc}
    \toprule
    Fornecedores Removidos & Veículos Conectados & Perda Acumulada \\
    \midrule
    0 & 36 & 0\% \\
    1 & 29 & 19\% \\
    2 & 27 & 25\% \\
    3 & 22 & 39\% \\
    4 & 22 & 39\% \\
    5 & 22 & 39\% \\
    \bottomrule
    \end{tabular}
\end{table}

A rede não degrada graciosamente — ela sofre uma \textbf{transição de fase abrupta}. A remoção do primeiro Hub já causa fragmentação massiva, indicando ausência de redundância.

\section{Aplicações Práticas: Predição e Otimização}

\subsection{Algoritmo de Recomendação e Padronização}
Utilizando a técnica de \textbf{Filtragem Colaborativa} baseada nos clusters detectados, desenvolvemos um algoritmo capaz de prever quais peças "faltam" no cadastro de um veículo:
\begin{quote}
Se 70\%+ dos carros do Cluster A usam a "Peça X", e o Veículo Y (membro do Cluster A) não a usa, o algoritmo sugere essa peça.
\end{quote}

Isso tem dupla utilidade: (1) Identificação de erros na base de dados e (2) Sugestão de oportunidades de padronização futura.

\begin{table}[H]
    \centering
    \caption{Oportunidades de Padronização Identificadas}
    \begin{tabular}{lp{9cm}}
    \toprule
    Veículo Alvo & Sugestão de Componente \\
    \midrule
    Honda CR-V & Sistema ABS Bosch \\
    Mercedes E-Class & Sistema ABS Bosch \\
    Ford Focus & Airbag de Cortina \\
    Ford Kuga & Sistema ABS Bosch \\
    \bottomrule
    \end{tabular}
\end{table}

\subsection{Backbone da Indústria (Árvore Geradora Máxima)}
Para entender a estrutura mínima necessária para manter a indústria conectada, calculamos a \textbf{Árvore Geradora Máxima (Maximum Spanning Tree)}. A Figura \ref{fig:backbone} mostra o "esqueleto" da indústria.

\begin{figure}[H]
    \centering
    \includegraphics[width=0.95\textwidth]{../assets/fig5_backbone.png}
    \caption{Árvore Geradora Máxima (backbone) da rede projetada. Cada aresta representa a conexão mais forte entre veículos, revelando a estrutura mínima de integração industrial.}
    \label{fig:backbone}
\end{figure}

\subsection{Quantificação da Eficiência de Estoque}
A motivação econômica foi quantificada comparando-se o número de SKUs necessários:
\begin{itemize}
    \item \textbf{Demanda agregada:} 140 partes (se cada carro tivesse peças exclusivas)
    \item \textbf{Inventário real:} 48 peças únicas
    \item \textbf{Redução de complexidade logística:} \textbf{65.7\%}
\end{itemize}
Este ganho de eficiência explica a adesão massiva das montadoras a este modelo, apesar dos riscos sistêmicos apontados.

\section{Limitações do Estudo}
\begin{enumerate}
    \item \textbf{Dataset representativo:} Embora cubra 36 veículos de múltiplas montadoras, não representa a totalidade do mercado global.
    \item \textbf{Pesos uniformes:} As arestas no grafo bipartido não diferenciam a criticidade individual de cada peça para cada veículo.
    \item \textbf{Dinamismo temporal:} A análise é estática; a evolução da rede ao longo do tempo não foi modelada.
\end{enumerate}

\section{Conclusão e Trabalhos Futuros}
A aplicação da Teoria dos Grafos à cadeia de suprimentos automotiva revelou uma topologia altamente otimizada para a eficiência, mas estruturalmente frágil:
\begin{enumerate}
    \item A rede possui uma estrutura de \textbf{Mundo Pequeno} e alta clusterização (Figura \ref{fig:clustering}), facilitando a difusão de falhas.
    \item A dependência de \textbf{Super-Hubs} (Figuras \ref{fig:centrality_comp} e \ref{fig:hubs}) cria riscos de colapso total em caso de falhas pontuais.
    \item A \textbf{homogeneização tecnológica} (Figura \ref{fig:mixing}, assortatividade $\approx 0$) indica que Premium e Economy compartilham a mesma base tecnológica.
    \item A \textbf{decomposição K-Core} (Figura \ref{fig:kcore}) revelou um núcleo denso de 27 conexões mínimas.
    \item A rede exibe comportamento \textbf{Scale-Free} (Figura \ref{fig:degree_dist}), vulnerável a ataques direcionados.
\end{enumerate}

\subsubsection*{Recomendações}
\begin{itemize}
    \item Gestores devem utilizar a análise de \textbf{K-Core} para monitorar a saúde dos ecossistemas de plataformas.
    \item Implementar \textbf{redundância estratégica} para os componentes identificados na Tabela \ref{tab:hubs}.
    \item Utilizar o algoritmo de recomendação para identificar oportunidades de padronização sem aumentar riscos.
\end{itemize}

\subsection{Trabalhos Futuros}
\begin{itemize}
    \item Modelagem temporal da evolução da rede
    \item Inclusão de pesos diferenciados por criticidade
    \item Simulações de Monte Carlo para análise probabilística de falhas
    \item Extensão para incluir fornecedores de segundo e terceiro nível
\end{itemize}

\end{document}